\documentclass[german]{alpication}


\usepackage[T1]{fontenc}
\usepackage[utf8]{inputenc}
\usepackage[ngerman]{babel}

% \renewcommand*\familydefault{\sfdefault} % use sans-serif font
% \definecolor{rulecolor}{HTML}{FF0000}    % change the color of horizontal lines
% \setlength{\rulewidth}{.5pt}             % change the thickness of these lines


	\author{Max Mustermann}
	\title{Bewerbung}

% 
% Informationen zum Bewerber
% 
	\applicantName		{Max \mbox{Mustermann}} % \mbox verhindert Umbrüche im Wort
	\applicantStreet	{Musterstraße 1b}
	\applicantCity		{12345 Musterstadt}
	\applicantPhone		{0123\,456\,78\,910}
	\applicantMail		{muster@mail.de}
	\applicantLinkedIn	{max-mustermann}	% optional; verlinkt zu linkedin.com/in/<whatever you typed>
	
	\applicantPosition	{Patroneur}		% steht im Lebenslauf als Untertitel unter dem Namen
	
	\applicantPicture	{examplePic.png}
	\applicantSign		{exampleSign.png}
	\signCity		{Musterstadt}
	
	% Optionale Informationen
	\applicantBirthday	{01.02.1934}
	\applicantBirthplace	{Musterstadt}
% 	\applicantFamilyStatus	{Ledig}
% 	\applicantNationality	{Deutsch}
	
	\toAddress{	Paul Personaler\\
				Muster GmbH\\
				Musterstraße 42\\
				12345 Musterstadt	}

\begin{document}

% 
% Das Anschreiben (optional)
% 
\begin{correspondence}{Bewerbung als Patroneur.}		% Betreffzeile
	\opening{Sehr geehrter Herr Personaler,}	% Anrede

	Lorem ipsum dolor sit amet, consetetur sadipscing elitr, sed diam nonumy eirmod tempor invidunt ut labore
	et dolore magna aliquyam erat, sed diam voluptua. At vero eos et accusam et justo duo dolores et ea rebum.
	
	Stet clita kasd gubergren, no sea takimata sanctus est Lorem ipsum dolor sit amet. Lorem ipsum dolor sit amet,
	consetetur sadipscing elitr, sed diam nonumy eirmod tempor invidunt ut labore et dolore magna aliquyam erat,
	sed diam voluptua. At vero eos et accusam et justo duo dolores et ea rebum. Stet clita kasd gubergren, no sea
	takimata sanctus est Lorem ipsum dolor sit amet.

	\closing{Mit freundlichen Grüßen}
	
	\appendixList	% optional: Auflistung der Anhänge, benötigt 2-fache Ausführung von LaTeX
	
\end{correspondence}

% 
% Motivationsschreiben (optional; manchmal für Bewerbungen auf Studienplätze und Praktika gefordert). Die Umgebung kann
% beliebig verschoben werden, zB hinter den Lebenslauf.
%
\begin{motivation}
	Stet clita kasd gubergren, no sea takimata sanctus est Lorem ipsum dolor sit amet. Lorem ipsum dolor sit amet,
	consetetur sadipscing elitr, sed diam nonumy eirmod tempor invidunt ut labore et dolore magna aliquyam erat,
	sed diam voluptua. At vero eos et accusam et justo duo dolores et ea rebum. Stet clita kasd gubergren, no sea
	takimata sanctus est Lorem ipsum dolor sit amet.
\end{motivation}


%
% Der Lebenslauf (optional)
%
% Nutzung: \makeCV{<linke Spalte>}{<rechte Spalte>}
% 
\makeCV{%

	% Linke Spalte
	\leftColumnPicture % Profilbild
	\leftColumnApplicantNameAndPosition % Name und aktuelle Position

	\begin{leftColumnItemize}{Schl\"usselkompetenzen}
		\item Gewebe
		\item Textilien
		\item Muster
	\end{leftColumnItemize}

	\begin{leftColumnItemize}{Referenzen}
		\item Mutter Theresa
		\item Helmut Kohl
	\end{leftColumnItemize}

	\begin{leftColumnItemize}{Sprachen}
		\item Deutsch (Muttersprache)
		\item English (Well well)
	\end{leftColumnItemize}

	\begin{leftColumnItemize}[label={}, itemsep=-12pt]{Kontakt}
		\item \aStreet
		\item \aCity \\[5mm]
		\item \Telefon\hspace{6pt}\aPhone
	    \item \Letter\hspace{7pt}\href{mailto:\aMail}{\texttt{\aMail}}
	\end{leftColumnItemize}

	% \begin{leftColumnItemize}[label={}, itemsep=-12pt]{Sonstiges}
	% 	\item \Lborn~\aBirthday
	% 	\item \Lin~\aBirthplace
	% \end{leftColumnItemize}

}{%

	% Rechte Spalte
	\begin{cvsection}{Schulbildung}
		\item [04$\vert$1956 -- 05$\vert$1960] 	Muster-Grundschule Musterstadt
		\item [07$\vert$1960 -- 05$\vert$1966] 	Städtische Realschule Musterstadt
	\end{cvsection}

	\begin{cvsection}{Ausbildung}
		\item [06$\vert$1966 -- 08$\vert$1969] 	Patroneursausbildung Stickerei Mustermacher,\\Spezialisierung auf Gewebe
		\item [07$\vert$1968]			Muster-Design-Workshop
	\end{cvsection}

} % Ende von \makeCV


% 
% Anhänge (optional)
% 
\appendixFolder{./appendix}
\append{Beispiel-Dokument}{example}	% \append{Name (für optionale Auflistung)}{Dateiname im appendixFolder}
%\append{...}{...}


\end{document}
